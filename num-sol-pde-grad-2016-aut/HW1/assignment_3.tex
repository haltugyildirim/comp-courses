%%%%%%%%%%%%%%%%%%%%%%%%%%%%%%%%%%%%%%%%%
% Short Sectioned Assignment
% LaTeX Template
% Version 1.0 (5/5/12)
%
% This template has been downloaded from:
% http://www.LaTeXTemplates.com
%
% Original author:
% Frits Wenneker (http://www.howtotex.com)
%
% License:
% CC BY-NC-SA 3.0 (http://creativecommons.org/licenses/by-nc-sa/3.0/)
%
%%%%%%%%%%%%%%%%%%%%%%%%%%%%%%%%%%%%%%%%%

%----------------------------------------------------------------------------------------
%	PACKAGES AND OTHER DOCUMENT CONFIGURATIONS
%----------------------------------------------------------------------------------------

\documentclass[paper=a4, fontsize=11pt]{scrartcl} % A4 paper and 11pt font size

\usepackage[T1]{fontenc} % Use 8-bit encoding that has 256 glyphs
\usepackage{fourier} % Use the Adobe Utopia font for the document - comment this line to return to the LaTeX default
\usepackage[english]{babel} % English language/hyphenation
\usepackage{amsmath,amsfonts,amsthm} % Math packages
\usepackage{adjustbox} %for adjusting the table size
\usepackage[affil-it]{authblk}  %for adjusting the author and affilition font size
\usepackage[font=scriptsize]{caption} %for adjusting the font size of captions
\usepackage{multirow}
\usepackage{hhline}

\usepackage{sectsty} % Allows customizing section commands
\allsectionsfont{\centering \normalfont\scshape} % Make all sections centered, the default font and small caps

\renewcommand\Authfont{\fontsize{12}{14.4}\selectfont}
\renewcommand\Affilfont{\fontsize{9}{10.8}\itshape}

\usepackage{fancyhdr} % Custom headers and footers
\pagestyle{fancyplain} % Makes all pages in the document conform to the custom headers and footers
\fancyhead{} % No page header - if you want one, create it in the same way as the footers below
\fancyfoot[L]{} % Empty left footer
\fancyfoot[C]{} % Empty center footer
\fancyfoot[R]{\thepage} % Page numbering for right footer
\renewcommand{\headrulewidth}{0pt} % Remove header underlines
\renewcommand{\footrulewidth}{0pt} % Remove footer underlines
\setlength{\headheight}{13.6pt} % Customize the height of the header

\numberwithin{equation}{section} % Number equations within sections (i.e. 1.1, 1.2, 2.1, 2.2 instead of 1, 2, 3, 4)
\numberwithin{figure}{section} % Number figures within sections (i.e. 1.1, 1.2, 2.1, 2.2 instead of 1, 2, 3, 4)
\numberwithin{table}{section} % Number tables within sections (i.e. 1.1, 1.2, 2.1, 2.2 instead of 1, 2, 3, 4)

\setlength\parindent{0pt} % Removes all indentation from paragraphs - comment this line for an assignment with lots of text

%----------------------------------------------------------------------------------------
%	TITLE SECTION
%----------------------------------------------------------------------------------------

\newcommand{\horrule}[1]{\rule{\linewidth}{#1}} % Create horizontal rule command with 1 argument of height

\title{	
\normalfont \normalsize 
\textsc{Istanbul Technical University, Mathematics Department \\ Numerical Methods for Partial Differential Equations I} \\ [25pt]
\horrule{0.5pt} \\[0.4cm] % Thin top horizontal rule
\large Homework I \\ % The assignment title
\horrule{0.5pt} \\[0.4cm] % Thick bottom horizontal rule
}

\author{Haydar Altu\u{g} Y{\i}ld{\i}r{\i}m \\ 509161108} 

\date{\normalsize\today} % Today's date or a custom date

\begin{document}

\maketitle % Print the title

%----------------------------------------------------------------------------------------
%	PROBLEM 1
%----------------------------------------------------------------------------------------

\section{Numerical and Analytical Derivation of Functions and Relative Errors}

Find the numerical(forward, backward and central) and analytical solutions of the functions $u_1=sin(x)$ and $u_2 = e ^{ {x^ 2}} $ at the point $x=1$ and calculate the percentage error.

%the table below is made with the help of the site; http://ericwood.org/excel2latex/ with addition of captions and adjustbox
\begin{table}[ht]
\centering
\begin{adjustbox}{width=1\textwidth}
\small
\begin{tabular}{ | l | l | l | l | l | l | l | l | l | l | l | l | l | }
\hline
	$u_1$ & \multicolumn{4}{c|}{forward differences}  & \multicolumn{4}{c|}{backward differences} &  \multicolumn{4}{c|}{central differences}   \\ \hline
	$\Delta$ x & analytic & numeric & error(\%) & control & analytic & numeric & error(\%) & control & analytic & numeric & error(\%) & control \\ \hline
	0.1 & 0.5403 & 0.4973 & 7.9585 & -0.9008 & 0.5403 & 0.5814 & 7.6068 & -0.8812 & 0.5403 & 0.5393 & 0.1665 & 0.7783 \\ \hline
	0.05 & 0.5403 & 0.519 & 3.9422 & -0.4579 & 0.5403 & 0.5611 & 3.8497 & -0.4499 & 0.5403 & 0.54 & 5.5524E-2 & 0.9650 \\ \hline
	0.01 & 0.5403 & 0.536 & 0.7958 & 4.9583E-2 & 0.5403 & 0.5444 & 0.7773 & 5.4692E-2 & 0.5403 & 0.5402 & 1.8508E-2 & 0.8663 \\ \hline
	5E-3 & 0.5403 & 0.5381 & 0.4071 & 0.1695 & 0.5403 & 0.5423 & 0.3886 & 0.1783 & 0.5403 & 0.5403 & 0 &  \\ \hline
	1E-3 & 0.5403 & 0.5397 & 9.2541E-2 & 0.3445 & 0.5403 & 0.5406 & 7.4032E-2 & 0.3768 & 0.5403 & 0.5403 & 0 &  \\ \hline
\end{tabular}
\end{adjustbox}
\caption{$u_1 = sin(x)$ calculations}
\end{table} 

\begin{table}[ht]
\centering
\begin{adjustbox}{width=1\textwidth}
\small
\begin{tabular}{ | l | l | l | l | l | l | l | l | l | l | l | l | l |}
\hline
	$u_2$ & \multicolumn{4}{c|}{forward differences}  & \multicolumn{4}{c|}{backward differences} &  \multicolumn{4}{c|}{central differences}   \\ \hline
	$\Delta$ x & analytic & numeric & error(\%) & control & analytic & numeric & error(\%) & control & analytic & numeric & error(\%) & control \\ \hline
	0.1 & 5.4364 & 6.352 & 16.8398 & -1.2263 & 5.4364 & 4.7037 & 13.4792 & -1.1296 & 5.4364 & 5.5278 & 1.6793 & -0.2251 \\ \hline
	0.05 & 5.4364 & 5.868 & 7.937 & -0.6914 & 5.4364 & 5.0503 & 7.1019 & -0.6543 & 5.4364 & 5.4592 & 0.4175 & 0.2915 \\ \hline
	0.01 & 5.4364 & 5.5191 & 1.5175 & -9.0567E-2 & 5.4364 & 5.3559 & 1.4825 & -8.5507E-2 & 5.4364 & 5.4374 & 1.6554E-2 & 0.8905 \\ \hline
	5E-3 & 5.4364 & 5.4775 & 0.7541 & 5.3252E-2 & 5.4364 & 5.395 & 0.7449 & 5.5568E-2 & 5.4364 & 5.4367 & 3.6788E-3 & 1.0579 \\ \hline
	1E-3 & 5.4364 & 5.4447 & 0.1508 & 0.2738 & 5.4364 & 5.4283 & 0.1489 & 0.2756 & 5.4364 & 5.4364 & 0 &  \\ \hline
\end{tabular}
\end{adjustbox}
\caption{$u_2 = e ^{ {x^ 2}} $ calculations} 
\end{table} 

\end{document}